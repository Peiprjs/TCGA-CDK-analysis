\documentclass{article}

\usepackage{Sweave}
\begin{document}
\Sconcordance{concordance:Test.tex:Test.Rnw:1 2 1 1 0 2 1 1 30 29 0 13 1 1 2 12 1 6 0 %
1 5 3 1}


\begin{Schunk}
\begin{Sinput}
> # Script:       Step1_DataExploration.R
> # Description:  In this script, we will explore the differential gene 
> #               expression dataset comparing lung cancer vs. healthy tissue
> #               samples. The RNA-sequencing dataset was retrieved from 
> #               TCGA (The Cancer Genome Atlas) and pre-processed in R. 
> #               Differential gene expression analysis was performed with the 
> #               DESeq2 R-package. 
> # Version: 2.0
> # Last updated: 2025-06-08
> # Author: mkutmon
> 
> # #############################################
> # R INSTRUCTIONS
> # #############################################
> 
> # * Lines that start with a # are comments
> # * You can run a code line by placing the cursor in the line and clicking 
> #   CTRL/Command + Enter
> # * In between you will see the ??? Questions which refers to a question 
> #   in the question document provided on Canvas.
> 
> 
> # #############################################
> # R SETUP
> # #############################################
> 
> # Here we install and load all required packages. 
> # This might take a few minutes
> if (!("BiocManager" %in% installed.packages())) { install.packages("BiocManager", update=FALSE) }
> if (!("rstudioapi" %in% installed.packages())) { BiocManager::install("rstudioapi", update=FALSE) }
> if (!("org.Hs.eg.db" %in% installed.packages())) { BiocManager::install("org.Hs.eg.db", update=FALSE) }
> if (!("dplyr" %in% installed.packages())) { BiocManager::install("dplyr", update=FALSE) }
> if (!("EnhancedVolcano" %in% installed.packages())) { BiocManager::install("EnhancedVolcano", update=FALSE) }
> if (!("readxl" %in% installed.packages())) { BiocManager::install("readxl", update=FALSE) }
> if (!("clusterProfiler" %in% installed.packages())) { BiocManager::install("clusterProfiler", update=FALSE) }
> if (!("enrichplot" %in% installed.packages())) { BiocManager::install("enrichplot", update=FALSE) }
> if (!("Rgraphviz" %in% installed.packages())) { BiocManager::install("Rgraphviz", update=FALSE) }
> if (!("RCy3" %in% installed.packages())) { BiocManager::install("RCy3", update=FALSE) }
> if (!("msigdbr" %in% installed.packages())) { BiocManager::install("msigdbr",update=FALSE) }
> if (!("RColorBrewer" %in% installed.packages())) { BiocManager::install("RColorBrewer",update=FALSE) }
> if (!("readr" %in% installed.packages())) { BiocManager::install("readr",update=FALSE) }
> if (!("rWikiPathways" %in% installed.packages())) { BiocManager::install("rWikiPathways",update=FALSE) }
> library(rstudioapi)
> library(org.Hs.eg.db)
> library(dplyr)
> library(EnhancedVolcano)
> library(readxl)
> library(clusterProfiler)
> library(enrichplot)
> library(Rgraphviz)
> library(RCy3)
> library(msigdbr)
> library(RColorBrewer)
> library(readr)
> library(rWikiPathways)
> 
> # We will set the working directory to the location where the current 
> # script is located. This way, we can use relative file path locations. 
\end{Sinput}
\end{Schunk}



\end{document}
